We consider the solution of discrete ill-posed problems with linear
forward models of the form
\begin{equation}
\label{eq:exact}
A x = b = \btrue + \eta,
\end{equation}
where $A \in \mathbb{R}^{m\times n}$ (with $m \geq n$) is the forward
operator, $\xtrue$ is the ``true'' solution that we look to recover,
$\btrue = A\xtrue$ is the ``true'' data generated by applying $A$
to $\xtrue$, and $\eta$ is an unknown additive white noise vector.  We
assume that both the true solution and noise-free data are unknown, so
that we can only access the problem via the forward operator, $A$, and
the vector, $b$.  We focus on imaging problems, where $x$ and
$\xtrue$ are taken to be the vectorized forms of digital images, and
$A$ has singular values that decay rapidly towards zero, with the
singular vectors corresponding to the largest singular values
representing smooth modes and those corresponding to the smallest
singular values representing high-frequency modes.  Due to the
presence of the noise and the decay of the singular values, na\"ive
solution of the linear system via least squares, as $A^TA x = A^Tb$,
will be contaminated by noise, with the extent of noise corruption
determined by the decay rate of the singular values as well as the
noise level \cite{Hansenbk}.  Moreover, if $\text{rank}(A)< n$, then
even without noise, there is no unique least-squares solution.

To combat the effects of the (near) rank deficiency and noise, it is typical to
compute an approximation to $\xtrue$ by augmenting the least-squares
formulation with a suitable
Tikhonov regularization \cite{Tikhonov} term, giving
\begin{equation}
\label{eq:tikhgeneric}
\min_{x} \| A x - b \|_2^2 + \lambda^2 \| L x \|_2^2
\end{equation}
where $L$ denotes a regularization matrix that is used to enforce an
expectation on the properties of the solution, $x$, and $\lambda$ is a
parameter used to control the ``balance'' in the minimization between
the two terms.  When $x$ is known to be smooth, a common choice of $L$
might be as the discrete gradient operator or discrete
Laplacian.  The proper choice of $\lambda$ is well-known to be a difficult task,
which can only be accomplished by solving
Equation \eqref{eq:tikhgeneric} for many different values of $\lambda$
and comparing the resulting solution set.

In this paper, we focus on the case of edge-preserving regularization
schemes \cite{CRVogel_2002a, Gazzola_etal_2020}, where the
Tikhonov regularization operator, $\|Lx\|^2_2$, is replaced by another
operator that allows sharp
changes in the values of the solution, $x$, but only when those
changes are deemed to form a ``natural edge'' in the recovered image.
Deciding on whether or not such a jump is an edge is itself a complex
task, particularly without assuming more information on the problem
structure than is preferable in a general-purpose algorithm.  Here, we
follow the \textit{iteratively reweighted norm} (IRN) method, proposed
in \cite{Gorodnitsky_Rao_1992} and further developed in
\cite{Rodriguez_Wohlberg, RARenaut_etal_2017}. In this approach, a non-quadratic
constraint, such as $\|Lx\|_p^p$ for $p\approx 1$ or the total
variation of $x$, $\mathrm{TV}(x)$, is reformulated as a sequence of
quadratic regularization problems,
\begin{equation}
\label{eq:sequence}
\min_{x} \| A x - b \|_2^2 + \lambda_\ell^2 \| M^{(\ell)} x \|_2^2,
\end{equation}
where $M^{\ell}$ is defined based on $L$ and the approximate solution
of the $(\ell-1)$th problem, and $\lambda_\ell$ is chosen based on
established approaches for the $\ell$th problem.

We follow the form of the ``inner-outer'' iteration for
edge preservation introduced in \cite{Gazzola_etal_2020}.  In this
approach, there are 3 steps to each outer iteration:
\begin{enumerate}
\item Updating $M^{(\ell-1)}$ to form $M^{(\ell)}$,
\item Choosing $\lambda_\ell$ to appropriately balance the data and
  regularization terms in \eqref{eq:sequence}, and
\item Solving the minimization problem in \eqref{eq:sequence} for the
  optimal solution, $x^{(\ell)}$.
\end{enumerate}
As in \cite{Gazzola_etal_2020}, we form the sequence of quadratic
regularization operators as diagonal scalings of the gradient matrix,
$L$, defining $M^{(\ell)} = D^{(\ell)}L$, where $D^{(\ell)}$ is a
diagonal matrix, itself formed by the update $D^{(\ell)} =
\text{diag}(d^{(\ell)})D^{(\ell-1)}$, where $\text{diag}(d^{(\ell)})$
denotes the diagonal matrix with entries given by those of the vector
$d^{(\ell)}$.  Again following \cite{Gazzola_etal_2020}, we form
$d^{(\ell)}$ by considering the solution at step $\ell-1$,
$x^{(\ell-1)}$, and normalizing the gradient of the solution, defining
\[
g^{(\ell-1)} = |Lx^{(\ell-1)}| / \|Lx^{(\ell-1)}\|_\infty,
\]
where the absolute-value in the numerator is meant to denote an
element-wise operation on the vector $Lx^{(\ell-1)}$.  Noting that the
largest entries in $g^{(\ell-1)}$ take value 1 and should correspond
to the dominant edges in image $x^{(\ell-1)}$, we then compute the
entries in the weighting vector
\[
d^{(\ell)}_i = 1 - \left(g^{(\ell-1)}_i\right)^p,
\]
for a suitable choice of $p$ (here taken to be $p=2$).

In \cite{Gazzola_etal_2020}, a hybrid regularization approach
\cite{Kilmer_Hanson_Espanol_2007} was used as an ``inner iteration''
to solve for both $\lambda_\ell$ and $x^{(\ell)}$ once $M^{(\ell)}$ is
determined.  In this approach, the bidiagonalization technique from
\cite{Kilmer_Hanson_Espanol_2007} is used to create a basis set,
$\{z^{(\ell)}_j\}_{j=1}^k$, such that
\[
AZ_k^{(\ell)} = U_{k+1}^{(\ell)}B_k^{(\ell)} \text{ and
}M^{(\ell)}Z_k^{(\ell)} = \hat{U}_k^{(\ell)}\hat{B}_k^{(\ell)},
\]
where $Z_k^{(\ell)}$ is the matrix with columns given by
$z^{(\ell)}_j$, $U_{k+1}^{(\ell)}$ and $\hat{U}_k^{(\ell)}$ are
orthogonal matrices, and $B_k^{(\ell)}$ and $\hat{B}_k^{(\ell)}$ are
bidiagonal matrices.  Restricting $x$ in \eqref{eq:sequence} to lie in
the range of $Z_k^{(\ell)}$, $x = Z_k^{(\ell)}w$, leads to the
$k$-dimensional approximation,
\begin{equation}
  \label{eq:sequence_hybrid}
\min_{w} \| B_k^{(\ell)}w - \beta_1 e_1 \|_2^2 + \lambda_\ell^2 \| \hat{B}_k^{(\ell)} w \|_2^2,
\end{equation}
where $\beta_1 e_1 = \left(U_{k+1}^{(\ell)}\right)^Tb$.  In
\cite{Gazzola_etal_2020}, a further iteration occurs where the value
of $k$ is computed experimentally.  Here, a fixed set of values for
$\lambda$ are considered, along with a sequence of values of $k$.  For
each value of $k$ (increasing from the smallest), the set of problems
in \eqref{eq:sequence_hybrid} are solved for each of the chosen values
of$\lambda$, and the L-curve for this solution set is considered.  The
final choice of $k$ is made by considering the corner of the L-curve,
and stopping the iteration in $k$ when the estimated corner value
stays constant.  While this is relatively efficient (since the
individual problems used to form the L-curve are all low-dimensional
and a short-term recurrence is used to form $Z_k^{(\ell)}$,
$B_k^{(\ell)}$, and $\hat{B}_k^{(\ell)}$), it still incurs a
substantial cost per iteration, since the basis in $Z_k^{(\ell)}$
depends on $M^{(\ell)}$ and, therefore, cannot be reused for multiple
outer iterations.  Here, we reconsider the solution of the set of
problems in \eqref{eq:sequence}, using more direct approaches to the
least-squares problems.

{\bf TODO:} Misha should read and correct the above!

