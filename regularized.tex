\documentclass{article}
\usepackage{amsmath,amsfonts,color, amsthm, ifpdf}

\ifpdf
 \usepackage[pdftex]{graphicx}
\else
  \usepackage[dvips]{graphicx}
\fi

% Margins for plain pagestyle
\oddsidemargin=0.0in
\textwidth=6.5in
\textheight=9.0in
\topmargin=-0.5in



\title{Notes on (A)MG for regularised problem}
\author{Matthias Bolten, Misha E. Kilmer, and Scott P. MacLachlan}

\begin{document}
\maketitle

\begin{abstract}
blah blah blah
\end{abstract}


\section{Introduction}



In this paper, we are interested in designing efficient solvers for the sequence of least regularized least squares problems
\begin{equation} \label{eq:sequence} \min_{x} \| K x - b \|_2^2 + \lambda^2 \| R^{(i)} x \|_2^2 \end{equation}
or equivalently
\[ \min_x \left\| \left[ \begin{matrix} K \\ \lambda R^{(i)} \end{matrix} \right] x - \left[ \begin{matrix} b \\ 0 \end{matrix} \right] \right\|_2 \]
or 
\[ (K^TK + \lambda^2 L^T (W^{(i)})^2 L ) x = K^T b \]
where 
\[ R^{(i)} = W^{(i)} L \] 
for $L$ the discrete gradient, and $W^{(i)}$ is a diagonal weighting, and we asume the ``best'' value of the regularization parameter for this $i$th problem, $\lambda^{(i)}_{*}$, is {\bf not} known a priori.   Such problems
frequently arise in situations where the weights would ideally be computed from the desired $x_{true}$, but as this is unknown the weights are recomputed adaptively from increasingly better estimates of the desired solution.    
In our numerical examples, we consider two types of weightings from recent literature:  one which is designed to give solutions with ``sparse'' gradients \cite{IRLS} and one which is designed to preserve edges \cite{KilmerEtalICIAM15}.  

We will consider only
one well-known heuristic for estimating $\lambda^{(i)}_*$ -- the L-curve criterion -- but our solvers would also be applicable in scenarios where other selection mechanisms are used.       
The L-curve \cite{OlearyHansen,etc} is a parametric plot of $( \log_10 \|K x_{\lambda} - b \|_2,\log_10 \| R^{(i)} x_{\lambda} \|_2)$ as a function of $\lambda$.   The general idea is that the point of maximum positive curvature occurs where there is a good balance between the effect of the regularization term and data-fidelity term.   (Forward reference to a plot in the paper).   

The obvious approach is to ``solve'' on of the equivalent forms of (\ref{eq:sequence}) for a few choices of $\lambda$ and then plot the curve.   If implemented naively, however, this is expensive because
   \begin{itemize}
      \item When $\lambda$ is small,   $\left[ \begin{matrix} K \\ \lambda R^{(i)} \end{matrix} \right] $ is likely to be ill-conditioned, as the emphasis is on the ill-conditioned top block.   Thus, many iterations are required for an {\it accurate} solution.
      \item When $\lambda$ is large, the emphasis is on the bottom block, but as $i$ increases, $R^{(i)}$ tends
      to be ill-conditioned due to more accurate diagonal weighting and near zero terms.
      \end{itemize}
      

Alternatively, hybrid iterative methods for the class of regularization operators we consider ($R^{(i)}$ rectangular with possible non-trivial null-space) we consider, in which $x$ is constrained to live in a k-dimensional subspace whose basis is generated by an iterative procedure for (\ref{eq:sequence}), have been proposed \cite{KilmerEspanolHansen,KilmerEtalICIAM15}.  Related methods, not immediately applicable for the case of non-rectangular $K$ and $L$ we consider here, are given in \cite{SilvaNagy15}.   Hybrid
methods have an advantage that, given a suitable $k$, the value of $\lambda_*^{(i)}$ can be chosen for the k-dimensional projected problem very cheaply, and only one subspace is needed/created for all values of $\lambda$.   The cost of the hybrid methods varies by algorithm, but matrix vector products with $K, R^{(i)}$ and their transposes are a necessary evil, and thus the minimum cost of any of these methods is $k$ times the sum of the costs for those products.   (MEK:  need to clean this up.  emphasize non-square A, L, means must use my appraoch, but and potential expense of our hybrid method and the fact other hybrids won't work unless you could transform to standard form, then you need A-weighted pseudoinverse which is very expensive per iteration.) 

Our goal in this paper is to modify the ``obvious'' approach so that it is provides a computationally viable algorithm.
Ingredients:
   \begin{itemize}
     \item AMG preconditioning when $\lambda$ is larger and favors the diffusion like term.   
     \item Non-exact solves with $\lambda$ small (possibly combined with preconditioner for $A^TA$?  ), just to know enough about the l-curve to find the corner
     \item Leveraging properites of the l-curves as functions of $i$ to prune $\lambda$
     \end{itemize} 


\section{Problem Description}

We consider the solution of the regularized problem
\begin{equation}
\label{eq:regularized}
\min_{x} \|Kx-b\|^2 + \lambda^2\|WLx\|^2,
\end{equation}
where the equation $Kx=b$ represents a ``data'' term, where $b$ is
(noisy) data coming from some imaging process represented by $K$, and
the second term is a regularization term.  Here, $L$ is a discrete
gradient matrix, and $W$ is a diagonal weighting matrix with weights
in the range of $[0,1]$.  An equivalent formulation comes from first
writing this as a rectangular block least-squares problem,
\[
\min_{x} \left\| \left[\begin{array}{c} K \\ \lambda
      WL\end{array}\right]x - \left[\begin{array}{c} b \\ 0 \end{array}\right]\right\|^2,
\]
and, then, consider the resulting normal equations,
\begin{equation}
\label{eq:normal}
\left(K^TK + \lambda^2 L^TW^2L\right)x = K^Tb.
\end{equation}
Since $L$ is a discrete gradient, the second matrix in this expression
corresponds to a variable-coefficient diffusion operator (with
diffusion coefficient related to the weights in $W$), while the first
matrix and right-hand side come from the standard normal equations for
$Kx=b$.

Overall, this comes from a nonlinear problem, where we wish to choose
$x$, $W$, and $\lambda$ to minimize this functional while doing
something (iteration to drive $\ell_2$ regularization towards $\ell_1$
regularization).  We consider the case of fixed $W$, in the context of
this outer iteration.  At each step of the outer iteration, we need to
choose a good regularization parameter, $\lambda$, and find the
solution, $x$, for that value of $\lambda$.

Describe outer iteration for choosing $W$, so that $\|WLx\|_2 \approx
\|Lx\|_1$, then middle iteration for choosing $\lambda$ for fixed $W$,
and L-curve criteria, then need for an inner iteration to solve for
$x$ given $W$ and $\lambda$.

\section{Algorithm}

For fixed $W$, the problem posed in \eqref{eq:regularized} can be
considered in 3 regimes.
\begin{enumerate}
\item When $\lambda > C \frac{\sigma(K)}{\sigma{WL}}$, where
  $\sigma(K)$ is the largest singular value of $K$ and $\sigma{WL}$ is
  the largest singular value of $WL$.  In this regime, the
  regularization (diffusive) term is dominant.
\item When $C \frac{\sigma(K)}{\sigma{WL}} > \lambda > c
  \frac{\sigma(K)}{\sigma{WL}}$, for $\mathcal{O}(1)$ constants $c$
  and $C$, the two terms in \eqref{eq:regularized} are {\it in
    balance}.
\item When $\lambda < c \frac{\sigma(K)}{\sigma{WL}}$, the data term
  in \eqref{eq:regularized} is dominant.
\end{enumerate}
For each of these cases, a different solution strategy is optimal.

In the first two cases, the presence of a significant diffusive term
in \eqref{eq:regularized} suggests the use of a multigrid approach.
Here, we make use of the equivalence between the formulations in
\eqref{eq:regularized} and \eqref{eq:normal}, designing an algorithm
that applies directly to the sparse matrices in
\eqref{eq:regularized}, but equivalent to constructing an algebraic
multigrid hierarchy \cite{} based on \eqref{eq:normal}.

Considering the matrix $A = L^TW^2L$, which corresponds to a
variable-coefficient diffusion operator on a two-dimensional grid,
classical algebraic multigrid (AMG) is well-known to be an efficient
algorithm for the solution of linear systems $Au=f$.  Here, we apply a
standard AMG setup phase to $A$ to create a hierarchy of interpolation
operators, $P$, and coarse-grid operators, $A_c = P^TAP$ (using
Galerkin coarsening).  From these interpolation operators, we also
create {\it one-sided} coarse-grid operators $K_c = KP$ and $(WL)_c =
WLP$, at all levels in the multigrid hierarchy.  This allows us to
write down the coarse-grid correction process for a multigrid
algorithm, by taking a current approximation, $\hat{x}$, and asking to
solve for a correction in the range of $P$:
\[
\min_{y_c}\left\| \left[\begin{array}{c} K \\ \lambda
      WL\end{array}\right](\hat{x}+Py_c) - \left[\begin{array}{c} b \\ 0 \end{array}\right]\right\|^2.
\]
Computing the minimum, we characterize the coarse-grid solution $y_c$
as the solution to
\[
\left(P^TK^TKP + \lambda^2P^TL^TW^2LP\right)y_c =
P^T\left[\begin{array}{c} K \\ \lambda WL\end{array}\right]^T
\left(\left[\begin{array}{c} b \\ 0 \end{array}\right] - \left[\begin{array}{c} K \\ \lambda
      WL\end{array}\right]\hat{x}\right),
\]
where the right-hand side is the restriction of the residual in the
normal equations associated with $\hat{x}$.  Using either of these to
pose the coarse-grid problem within a multigrid cycle, all that
remains is to specify the multigrid relaxation scheme and cycling
parameters.  In all that follows, we consider only multigrid V-cycles,
as W-cycles (or F-cycles) do not appear to be beneficial.

In the diffusion-dominated regime, the solution of
\eqref{eq:regularized} is primarily determined by the diffusive term
and, as such, we propose a multigrid smoother that is appropriate for
this term.  While classical AMG applied to \eqref{eq:normal} would
typically use a Gauss-Seidel relaxation scheme, this is not
appropriate here, since it would require knowledge of the (dense)
lower-triangular part of $K^TK$.  Instead, we use a red-black-ordered
Jacobi iteration, which allows us to make use of efficient
matrix-vector products with $K$ and $WL$ and their transposes.
Following reduction-based multigrid ideas \cite{}, we use only a
single sweep of pre-relaxation, in a CF ordering, first computing
updates to the coarse-grid points, then to those points on the fine
grid that are not directly represented on the coarse grid.  For each
of these sub-sweeps, we compute a full residual of the rectangular
form of the system,
\[
\hat{r} = \left[\begin{array}{c} b \\ 0 \end{array}\right] -
\left[\begin{array}{c} K \\ \lambda WL\end{array}\right]\hat{x}
\]
but then compute a Jacobi-like update at only the points, $j$, under
consideration, as
\[
\hat{x}_j + {\left(\left[\begin{array}{cc} K^T & \lambda
      L^TW\end{array}\right]\hat{r}\right)_j}/{ \left(K^TK +
  \lambda^2L^TW^2L\right)_{jj}}.
\]
We note that this requires computing only part of the matrix-vector
product with $\left[\begin{array}{cc} K^T & \lambda
    L^TW\end{array}\right]$ and the diagonal entries of $K^TK +
\lambda^2L^TW^2L$.  With this approach, a CF sweep of relaxation costs
the same as 3/2 matrix-vector products with the normal equations (when
done implicitly using the sparse matrices $K$ and $WL$).

Numerical experiments below show that this is an efficient relaxation
scheme for the problem, leading to good multigrid convergence, only
for sufficiently large $\lambda$.  In the second case, when the two
terms are more in balance, the multigrid approach can still be
effective, but the CF relaxation scheme is no longer appropriate,
since the diffusion term is not dominant.  Here, to better reflect the
impact of $K$ on the linear system, we use the CG algorithm as a
relaxation scheme, taking five steps of CG on the normal equations
(with no preconditioner).  Probably need some references here about
using CG as a smoother - Wim's papers on Helmholtz + GMRES, Randy Bank
paper, Xuejun Xu cascadic MG paper???...

We note that the interpolation and coarse-grid operators used above do
not depend on $\lambda$ and, thus, can be formed once for each choice
of $W$ and reused for all values of $\lambda$ for which a multigrid
iteration is needed.

When $\lambda$ is small compared to $\frac{\sigma(K)}{\sigma(WL)}$,
then the problem is essentially equivalent to the unregularized
problem $min_x \|Kx-b\|^2$.  In this case, ... Here, we rely on the
fact that the outer (nonlinear) iteration is aiming to choose $W$ so
that the two terms are in balance.  Thus, while it is important to get
a sufficiently accurate solution when $\lambda$ is small to drive the
nonlinear iteration, we are not concerned with getting high levels of
accuracy in this case (since it occurs only early in the outer
(nonlinear) iteration).  In this case, there is little expected
benefit from a coarse-grid correction based on the diffusive term, so
we consider only a simple iteration.  We use unpreconditioned CG on
the normal equations, allowing up to 50 steps.  In practice, this
yields relatively little improvement on using the solution from the
next-largest $\lambda$ value as the initial guess, and the improvement
to be gained decreases as $\lambda$ does.

{\bf Ask Misha if there's something more reasonable to be done in the
  case of smallest values of $\lambda$.}


\section{Numerical Results}

\subsection{Deblurring?}

\subsection{Radon Transform}

Current example.  Play with noise level (once L-curve selection is
back in).

\subsection{Another example?}

\section{Conclusions}

\bibliographystyle{siam} \bibliography{regularized}


\end{document}
