Multigrid methods are optimal solvers for systems arising from the
discretization of elliptic PDEs. Geometric multigrid methods make use of the
following observation: When an iterative method like Gauss-Seidel is
applied to a linear system
\[
  A x = b, A \in \mathbb{R}^{n \times n}, x, b \in \mathbb{R}^n,
\]
arising from the discretization of a PDE, the reduction of the error
is relatively slow and depends on the discretization parameter
$h$. Nevertheless, the error $e = A^{-1} b - x$ is smooth after only a
few iterations of the iterative method. Obviously, a smooth error can
be represented well on a coarser mesh. Using the definition of the
residual $r = b - A x$ we can obtain the error as the solution of
$A e = r$. If a coarse representation $A_c$ of $A$ and $r_c$ of $R$ is
available, the system $A_c e_c = r_c$ can be solved instead and the
solution is less expensive. The restriction of $r$ to $r_c$ can be
obtained by simple injection or by more complicated schemes like,
e.g., full-weighting. The coarse grid operator $A_c$ is obtained by
using a rediscretization of the continuous operator on the coarser
mesh, or by using a variational principle resulting in the Galerkin
operator. The error $e_c$ on the coarse mesh is then interpolated to
the fine one, yielding an approximation $\tilde{e}$. Finally, the
current approximate solution can be updated as $x = x + \tilde{e}$. As
the interpolation process introduces high frequency error components
in the approximate solution, in most cases a couple of additional
smoothing steps is applied. The resulting method is called
\emph{two-grid} method. The idea can be applied recursively on level
$\ell$ to solve the coarse system $A_{\ell+1} e_{\ell+1} = r_{\ell+1}$
using $\gamma$ iterations of the described method on the subsequent
level $\ell+1$, resulting in a multigrid method. For given smoothers
$\mathcal{S}_\ell$ and $\tilde{\mathcal{S}}_\ell$,
$\ell = 0,\dots,\ell_\text{max}$, $\nu_1$ and $\nu_2$ pre- and
post-smoothing steps and restriction and prolongation operators
$R_\ell$ and $P_\ell$, $\ell = 1,\dots,\ell_{\text{max}-1}$ the
multigrid algorithm can be found in Algorithm~\ref{alg:mg}.

\begin{algorithm}
  \caption{Multigrid cycle $x_\ell = \mathcal{MG}_\ell(x_\ell,b_\ell)$}
  \label{alg:mg}
  \begin{algorithmic}
    \STATE{$x_\ell \leftarrow
      \mathcal{S}_\ell^{\nu_1}(x_\ell,b_\ell)$}
    \STATE{$r_\ell \leftarrow b_\ell - A_\ell
      x_\ell$}
    \STATE{$r_{\ell+1} \leftarrow R_\ell r_\ell$}
    \STATE{$e_{\ell+1} \leftarrow 0$}
    \IF{$\ell+1 = \ell_\text{max}$}
    \STATE{$e_{\ell_\text{max}} \leftarrow
      A_{\ell_\text{max}}^{-1} r_{\ell_\text{max}}$}
  \ELSE
%  \FOR{$j = 1,\dots,\xi$}
  \STATE{$e_{\ell+1} \leftarrow
    \mathcal{MG}_{\ell+1}(e_{\ell+1},r_{\ell+1})$}
%  \ENDFOR
  \ENDIF
  \STATE{$e_\ell \leftarrow P_\ell e_{\ell+1}$}
  \STATE{$x_\ell \leftarrow x_\ell +
    e_\ell$}
  \STATE{$x_\ell \leftarrow
    \tilde{\mathcal{S}}_\ell^{\nu_2}(x_\ell,b_\ell)$}
  \end{algorithmic}
\end{algorithm}

In case that the coarsening is not easily determined geometrically, an
algebraic multigrid method (AMG) can be used. For a given smoother
classical AMG automatically selects coarse mesh points and the
necessary restriction and prolongation operators. AMG for symmetric
problems usually chooses $R_\ell = P_\ell^T$, the coarse grid operator
is chosen to be the Galerkin coarse grid operator given by
$A_{\ell+1} = P_\ell^T A_\ell P_\ell$.

As noted before the second matrix $L^T W^2 L$ in \eqref{eq:normal}
corresponds to a variable-coefficient diffusion operator on a
two-dimensional grid. This can be solved efficiently using algebraic multigrid.

The first matrix in \eqref{eq:normal} corresponds to the operator that
has to be inverted. Usually, this is an integral operator, that is not
treated properly by algebraic multigrid. As the effective ratio
between the first and the second matrix in the system to be solved
depends on $\lambda$ we base our solver choice on the size of lambda
relative to the size of the singular values of $K^T K$ and $L^T W^2
L$.

For fixed $W$, the problem posed in \eqref{eq:regularized} is considered in 3 regimes:
\begin{enumerate}
\item When $\lambda > C \frac{\sigma(K)}{\sigma{WL}}$, where
  $\sigma(K)$ is the largest singular value of $K$ and $\sigma{WL}$ is
  the largest singular value of $WL$.  In this regime, the
  regularization (diffusive) term is dominant.
\item When $C \frac{\sigma(K)}{\sigma{WL}} > \lambda > c
  \frac{\sigma(K)}{\sigma{WL}}$, for $\mathcal{O}(1)$ constants $c$
  and $C$, the two terms in \eqref{eq:regularized} are {\it in
    balance}.
\item When $\lambda < c \frac{\sigma(K)}{\sigma{WL}}$, the data term
  in \eqref{eq:regularized} is dominant.
\end{enumerate}

In the first two cases, the diffusive term is significant, so the use
of a mulitigrid method is viable. We construct a multigrid method for
the regularized problem \eqref{eq:regularized} by reducing the size of
our solution, only. In terms of the block least-squares formulation
\eqref{eq:block-ls} the coarse grid correction corresponds to finding
a correction in the range of $P$ minimizing
\[
\min_{y_c}\left\| \left[\begin{array}{c} K \\ \lambda
      WL\end{array}\right](\hat{x}+Py_c) - \left[\begin{array}{c} b \\ 0 \end{array}\right]\right\|^2.
\]
Due to the equivalence of the least-squares problem and the solution
of the normal equations \eqref{eq:normal} the coarse-grid solution $y_c$ is then characterized as the solution to
\[
\left(P^TK^TKP + \lambda^2P^TL^TW^2LP\right)y_c =
P^T\left[\begin{array}{c} K \\ \lambda WL\end{array}\right]^T
\left(\left[\begin{array}{c} b \\ 0 \end{array}\right] - \left[\begin{array}{c} K \\ \lambda
      WL\end{array}\right]\hat{x}\right),
\]
where the right-hand side is the restriction of the residual in the
normal equations associated with $\hat{x}$.

The equivalence of the formulations allows to design a multigrid
algorithms for the normal equations \eqref{eq:normal}, and apply it
directly to the sparse matrices in \eqref{eq:regularized}. As $K^TK$
can get dense, this results in a reduction of compute time, further
efficient implementations of the application of $K$, e.g., using fast
transformations, can be used directly.

To obtain $P$, only the matrix $A = L^T W^2 L$ is considered. We apply a
standard AMG setup phase to $A$ to create a hierarchy of interpolation
operators, $P$, and coarse-grid operators, $A_c = P^TAP$ (using
Galerkin coarsening).  From these interpolation operators, we also
create the \textit{one-sided} coarse-grid operators $K_c = KP$ and $(WL)_c =
WLP$, at all levels in the multigrid hierarchy. All that remains is to
specify the multigrid relaxation scheme and cycling parameters.  In
all that follows, we consider only multigrid V-cycles, as W-cycles (or
F-cycles) do not appear to be beneficial.

In the diffusion-dominated regime, the solution of
\eqref{eq:regularized} is primarily determined by the diffusive term
and, as such, we propose a multigrid smoother that is appropriate for
this term.  While classical AMG applied to \eqref{eq:normal} would
typically use a Gauss-Seidel relaxation scheme, this is not
appropriate here, since it would require knowledge of the (dense)
lower-triangular part of $K^TK$.  Instead, we use a red-black-ordered
Jacobi iteration, which allows us to make use of efficient
matrix-vector products with $K$ and $WL$ and their transposes.
Following reduction-based multigrid ideas \cite{}, we use only a
single sweep of pre-relaxation, in a CF ordering, first computing
updates to the coarse-grid points, then to those points on the fine
grid that are not directly represented on the coarse grid.  For each
of these sub-sweeps, we compute a full residual of the rectangular
form of the system,
\[
\hat{r} = \left[\begin{array}{c} b \\ 0 \end{array}\right] -
\left[\begin{array}{c} K \\ \lambda WL\end{array}\right]\hat{x}
\]
but then compute a Jacobi-like update at only the points, $j$, under
consideration, as
\[
\hat{x}_j + {\left(\left[\begin{array}{cc} K^T & \lambda
      L^TW\end{array}\right]\hat{r}\right)_j}/{ \left(K^TK +
  \lambda^2L^TW^2L\right)_{jj}}.
\]
We note that this requires computing only part of the matrix-vector
product with $\left[\begin{array}{cc} K^T & \lambda
    L^TW\end{array}\right]$ and the diagonal entries of $K^TK +
\lambda^2L^TW^2L$.  With this approach, a CF sweep of relaxation costs
the same as 3/2 matrix-vector products with the normal equations (when
done implicitly using the sparse matrices $K$ and $WL$).

Numerical experiments below show that this is an efficient relaxation
scheme for the problem, leading to good multigrid convergence, only
for sufficiently large $\lambda$.  In the second case, when the two
terms are more in balance, the multigrid approach can still be
effective, but the CF relaxation scheme is no longer appropriate,
since the diffusion term is not dominant.  Here, to better reflect the
impact of $K$ on the linear system, we use the CG algorithm as a
relaxation scheme, taking five steps of CG on the normal equations
(with no preconditioner).  Probably need some references here about
using CG as a smoother - Wim's papers on Helmholtz + GMRES, Randy Bank
paper, Xuejun Xu cascadic MG paper???...

We note that the interpolation and coarse-grid operators used above do
not depend on $\lambda$ and, thus, can be formed once for each choice
of $W$ and reused for all values of $\lambda$ for which a multigrid
iteration is needed.

When $\lambda$ is small compared to $\frac{\sigma(K)}{\sigma(WL)}$,
then the problem is essentially equivalent to the unregularized
problem $min_x \|Kx-b\|^2$. As the outer (nonlinear) iteration is
aiming to choose $W$ such that both terms are in balance, this case is
not the focus of this work. Nevertheless, to drive the nonlinear
iteration a sufficiently accurate solution is required for small
$\lambda$, as well. As this case occurs mainly early in the outer
iterations, the accuracy does not have to be very high, though. To
obtain a reasonable good approximation we use CG without
preconditioner on the normal equations here, allowing up to 50
iterations. Using the solution of the next-largest $\lambda$ value as
the initial guess yields very little improvement, and the improvement
to be gained decreases as $\lambda$ does.

% {\bf Ask Misha if there's something more reasonable to be done in
% the case of smallest values of $\lambda$.}
