In this section, we reexamine the inner-outer iteration scheme from
\cite{Gazzola_etal_2020} with the goal of reducing its overall cost.
There are two factors that drive the cost of this scheme.  First of
all, there is the expensive computation of the joint bidiagonalization
of $A$ and $M^{(\ell)}$, which essentially amounts to applying an
unpreconditioned Krylov method to the problem, aiming to compute a
``large enough'' basis in order to be able to use the resulting space
to approximate solutions of \eqref{eq:sequence} for all values of
$\lambda_\ell$ to be sampled.  Secondly, there is the reconstruction
of approximate solutions for all values of $\lambda_\ell$ to be
sampled once a suitable approximation space is created.  To address
the first issue, we introduce a multigrid algorithm for the solution
of problems as in \eqref{eq:sequence}.  To address the second, we
introduce two key improvements in the selection of $\lambda_\ell$, by
solving over fewer potential values for the regularization parameter
and by using the iteration to introduce effective initial guesses for
the necessary linear solves.


