
We consider the solution of the regularized problem
\begin{equation}
\label{eq:regularized}
\min_{x} \|Kx-b\|^2 + \lambda^2\|WLx\|^2,
\end{equation}
where the equation $Kx=b$ represents a ``data'' term, where $b$ is
(noisy) data coming from some imaging process represented by $K$, and
the second term is a regularization term.  Here, $L$ is a discrete
gradient matrix, and $W$ is a diagonal weighting matrix with weights
in the range of $[0,1]$.  An equivalent formulation comes from first
writing this as a rectangular block least-squares problem,
\begin{equation}
  \label{eq:block-ls}
  \min_{x} \left\| \left[\begin{array}{c} K \\ \lambda
                           WL\end{array}\right]x - \left[\begin{array}{c} b \\ 0 \end{array}\right]\right\|^2,
\end{equation}
and, then, consider the resulting normal equations,
\begin{equation}
\label{eq:normal}
\left(K^TK + \lambda^2 L^TW^2L\right)x = K^Tb.
\end{equation}
Since $L$ is a discrete gradient, the second matrix in this expression
corresponds to a variable-coefficient diffusion operator (with
diffusion coefficient related to the weights in $W$), while the first
matrix and right-hand side come from the standard normal equations for
$Kx=b$.

Overall, this comes from a nonlinear problem, where we wish to choose
$x$, $W$, and $\lambda$ to minimize this functional while doing
something (iteration to drive $\ell_2$ regularization towards $\ell_1$
regularization).  We consider the case of fixed $W$, in the context of
this outer iteration.  At each step of the outer iteration, we need to
choose a good regularization parameter, $\lambda$, and find the
solution, $x$, for that value of $\lambda$.

Describe outer iteration for choosing $W$, so that $\|WLx\|_2 \approx
\|Lx\|_1$, then middle iteration for choosing $\lambda$ for fixed $W$,
and L-curve criteria, then need for an inner iteration to solve for
$x$ given $W$ and $\lambda$.
